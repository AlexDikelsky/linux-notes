\documentclass{article}
%\input inputed
%Not sure how input works yet

\begin{document}
A ``SDASDf''

hypen = -
en-dash = --
em-dash = ---
minus sign = $-$

\noindent{}Start of words $$Math in middle$$ end

Donald~E.~Knuth, ``Mathematical typography,'' {\it Bull.~Amer.~Math.~Soc.} {\bf 1} (1979), 337--372.
Type: 
Alice said, ``I always use an en-dash
instead of  hyphen when specifying page
numbers like `480--491' in a 
bibliograhy.''

\hbox{Some words}
\hbox{Some words}
\vbox{\hbox{Two lines}\hbox{Of type}}
\hrule
\vskip3in

test = ----
This produces an em dash and a hyphen

Lingautures test\newline
    find\newline
    AV\newline
    fluffier\newline

notice that the letters in fluffier are connected

Accent
\'o
Omlunte
\"o
different accent
\`a

\TeX\newline
\LaTeX\newline

the logo `\TeX'.

to be \bf bold or to \sl emphasize \rm something.

\tt words more words \bf boded words

also bolded\rm

\rm roman roman {\bf bolded} still roman

Ulrich Dieter, {\sl Journal f\"ur die reine und angewandte mathematic}\/  {\bf 201} (1959), 37--70.

{\sl explain how to type set a\/} roman {\sl word in the mittle of a sentence}

%Not sure why these cause an undefined control seq, they are legal on page 15 of the TexBook

%\tenrm this is in 10 point or
%\ninesl for slanted in 9 point font

This gets shelfful to remove the ff lingature

shelf{}ful

abc\H{o}sd

na\"\i{}ve

{\AE}sop's {\OE}uvres en fran\c{c}ais

{\sl Commentarii Academi{\ae} scientarum
imperialis petropolitan{\ae}\/} is now
{\sl Akademi{\t{\i}a} Nauk SSSR, Doklady}.

Ernesto Ces{\`a}ro

P{\`a}l Erd\H{o}s

{\O}ystein Ore

Stanis{\l}aw {\'S}wierczkowski

Serge{\u{\i}} {\t{Iu}}r'ev

Muhammad ibn M{\^u}s{\^a} al-Khw{\^a}rizm{\^i}

{\it \$}

\centerline{This should be {\it centered}.}
\centerline{\TeX\  has groups}
\centerline{This should be {centered}.}
\centerline{So should this}

%Procter & Gample's stock climbed to $2, a 10% gain.

\end{document}
