%List of catcodes: {{{
%   0   escape        \
%   1   begin         {
%   2   end           }
%   3   math shift    $
%   4   align tab     &
%   5   EOL           EOL
%   6   param         #
%   7   SUperscript   ^
%   8   Subscript     _
%   9   Ignored char  <null>
%   10  Space         \space
%   11  Letter	  [a-zA-Z]
%   12  Other char    none of the above
%   13  active char   ~
%   14  Comment char  %
%   15  Invalid char  <del>
%}}}

%This creaets a lot of errors
%Procter & Gamble's Stock climbed to $2, a  10% gain

%Ex 7.2
%\\ isn't mapped to literal \s because it's easy to remap it to something more useful this way

%TeX reads stuff as a stream of tokens
%Tokens are either a single char, or a control seq.
%When the following is read in, TeX reads it as
%   {\hskip 36pt}
%it reads it as
%   {₁  HSKIP  3₁₂   6₁₂   _₁₀  p₁₁    t₁₁  }₂

%You can define your own catcodes like this


{
\catcode`\<=1 
\catcode`\>=2
Words <\it this part is in italics delimited by angle brackets>
}   %You can end this with either a } or a > because both send catagory code 2
Stuff after. Here < and > have their normal wacky meanings

%This produces ``TeX, replacing the \ for ``.
\string\TeX

You can do stuff like \csname TeX\endcsname
to construct commads from lists of catagory codes. That lets you type
$\backslash$\TeX  like this  \csname \string TeX\endcsname

so \string ~ sdf  %This makes a normal tilde character

\catcode`\?=14
?You can do wacky stuff with this, like making this line a comment
\catcode`\!=3
!5 Math x+2={3\over{y}}5!

\end
