%the \char macro lets you type stuff in terms of ascii values

\char98 u\char98 \char98 le  %Types bubble but weirdly
\char123

%Ex 8.1
%   \def\%{\char`\%}
%works rather than
%   \def\%{\char`%}
%because the second % is treated as a comment rather than a character

You can also type styff with a bunch of carrot chars like
^^_ or ^^7 or ^^ff.
%This makes char ff one of the math symbols
\catcode`^^ff=3

^^ff sasf ^^ff

%You can also mess around with 
% and ^I with ^^M and ^^I

%Tex reads a sequence of lines
%The computer is always oin ne of these states:
%   State N	Start of line
%   State M	Middle of line
%   State S	Skipping blanks
%Not the same as modes
%States: eyes and mouth
%Modes : gastro-intestional tract

%TeX deletes and <space> chars at the right of an input line
%Then it inserts a <cr>

%If it sees a catcode=0, it reads. If it sees:
    %an empty name (\csname\endcsname) 
	%state -> M

    %not in cat 11 -> just that symbol
	%state -> depends

    %a bunch of letters -> go until the next nonletter
	%This becomes a cs token, like hbox in the example earlier
	%Then state S

%If it sees catcode=7 (superscript), and next char is the same thing:
    %if next thing is a num < 128, they are deleted & 64 is added to c
    %If the next thing is hex, all chars are repalced with that value

%If it sees anything not 0, 5, 7, 9, 10, 14, 15
    %or a normal catcode=7, then
	%Convert it to a token
	%Go to state M

%catcode = 5
%   Throw everything away. If already in N, \par
%   Otherwise, consider it a space

%catcode = 9
    %Ignore it

%EX 8.2
% a)  what is the difference between cat 5 and cat 14?
%	cat 5 -> if theres 2 in a row, insert a \par, which cat 14 doesnt do
%	cat 14-> throws away the rest of the line always
% b)  What is the difference between 3 and 4?
%	Nothing in this case, 
% c)  what is the difference between 11 and 12?
%	11 can be part of a command seq, while 12 cannot
% d)  are spaces ignored after active chars?
%	No
% e)  when a line ends with a comment, are spaces ignored at the start of the next line?
%	Yes, by defnintion because it will be in state N
% f)  can an ignored char be in the middle of a cs?
%	No

%EX 8.3
%When tex runs into an undef cs, it is in state S because
%you can have as many blanks as you want between \cs and the args
    %\vship 1in
%it was about to read the 1

%EX 8.4
%Assume that normal cat codes are used, except
%^^A -> 0
%^^B -> 7
%^^C -> 10
%^^M -> 11
%What is produced by
    %^^B^^BM^^A^^B^^C^^M^^@\M 
\catcode`^^A=0
%\catcode`^^B=7
\catcode`^^C=10
\catcode`^^M=11
%^^B%^^BM^^A^^B^^C^^M^^@\M 


%Example question
\catcode`i=7    %Rebinds i as the superscript char
\catcode`^^f9=3

%i^f9 sdf iif9  %Can't do this, the i^ breaks stuff
iif9 abc iif9

\end
