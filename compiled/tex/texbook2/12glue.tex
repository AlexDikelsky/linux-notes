%When you make a series of boxes, glue is put between them
%Glue has 3 parameters:
%   space   (ideal size)
%   stretch (coefficent by which to stretch)
%   shrink   (minimum size)
%Example:

%  _____
% |     |          ______           ___
% |     |    9    |      |    9    |   |      12     ________ 
% |     |+++++++++|      |+++++++++|   |++++++++++++|        |
% |     |         |      |         |   |            |        |
% |_____|         |______|         |___|            |        |
%   5     st=3       6      st=6     3     st=0     |________|
%	  sh=1              sh=2           sh=0          8   

%The ideal length is
    %5+9+6+9+3+12+8 = 52
%where you just add the lengths and the space of glue
%If you want to make the len=58, you:
    %1 find the total stretchability of the line  (in this case 3+6, or 9)
    %2 find how much space needs to be covered (new len - ideal len) (58-52, or 6)
    %3 for each glue, do
    %	    sp + (total_width / total_stretch) * st
%
%We have \hfil and \hfill to fill space, along with \vfil and \vfill
%You can tell the compiler to stop complaining with \hss, which means 
%make glue that can both shrink and stretch infinately. \vss does simular things
%You can negate with \vfilneg or \hfilneg

%Specify glue by typeing
%\vskip 2pt plus 3pt minus 1pt
%       ^       ^        ^
%    space   stretch   shrink

\line{Flush left\hfil}
\line{\hfil flush right}
\line{\hfil centered\hfil}
\line{some left\hfil osome right}

%EX 12.2
\line{\hfil\hfil Where? \hfil}  %This makes something a third of the way to the right
\line{\hfill\hfil where?\hfil}  %flush right

%EX 12.3
\def\centera#1{\line{\hfil#1\hfil}}
\def\centerb#1{\line{\hfill#1\hfill}}
\def\centerc#1{\line{\hss#1\hss}}

\centera{words}
\centerb{words}
\centerc{words}

%You can do stuff like
%\vskip 0pt plus 1fill

%\vfil is the same as \vskip 0pt plus 1fil

%EX 12.4
Mr.~\& Mrs.~User were married by Rev.~Drofnats,
who preached on Matt.~19\thinspace:\thinspace3--9.

%EX 12.5
Donald~E.\ Knuth, ``Mathematical typography,'' {\sl Bull.\ Amer.\ Math.\ Soc.\ {\bf 1}} (1979), 337--372.
%You technicly don't need the `\ ` after E because punctuation after capital letters is considered to be not a full stop char

%EX 12.6
lasd \hbox{A}. dsf
%to make sure the period kerning is correct

\tolerance=10000

\hbox to 103pt{``a b c d e''}   %This specifies the size of the hbox
\hbox to 13pt{``a b c d e''}    

%Specify exact spreading and shrinking
\hbox spread 123pt{``words that get spread out a lot''}
\hbox spread 32pt{``words that get spread out a lot''}
\hbox {``words that get spread out a lot''}

%EX 12.8
%box1 = 1pt high, 1pt deep, 1pt wide
%box2 = 2pt high, 2pt deep, 2pt wide
    %\setbox3=\hbox to3pt{\hfil\lower3pt\box1\hskip-3pt plus3fil\box2}
    %Find height, width and depth of box 3, and position of 1 and 2 with respect to box 3
%width=3
%hfil just makes stuff on the right of the 3pt
%lower makes it below the line]
%it then places box 1 there
%then it goes back a few spaces, and has glue to put box2
%box _+

%\vbox stuff stacks it verticly because it would look bad if line sizes 
%depended on the height of the letters
%You can use \baselineskip, \lineskip, and \lineskiplimit
%to make it look nice
%\baselineskip and \lineskip take a glue as an argument, while
%\lineskip takes a dimension.

%TeX tries to make every line use baselineskip
%   However, if that glue value causes the distance to be lower than the \lineskiplimit,
%   \lineskip is used instead.

%\lineskiplimit=123pt
%\showthe\lineskiplimit

\baselineskip=12pt plus 2pt
\lineskip=3pt minus 1pt
\lineskiplimit=2pt
%Generally you should only let \baselineskip have stretch values
%if you're doing a one page document
asdfsg sdf sdf sd gdf g d fgd us sfusidufhisudg sd fsdgsd sdyyk
asdfsg sdf sdf sd gdf g d fgd us sfusidufhisudg sd fsdgsd sdyyk
asdfsg sdf sdf sd gdf g d fgd us sfusidufhisudg sd fsdgsd sdyyk
asdfsg sdf sdf sd gdf g d fgd us sfusidufhisudg sd fsdgsd sdyyk

\end

