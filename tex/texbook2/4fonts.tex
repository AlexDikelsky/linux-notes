\sl Slanted \rm Roman \it italics \tt typewroter \bf bold \rm

%Slanted is just slanted letters, but \it actually changes the letters themselves
%Use grouping rahter than raw \tt\ s though, thorugh {\sl sdf}

%Ex 4.1
Ulrich Dieter, {\sl Journal f\"ur die reine und angewandte Mathematik} {\bf 201} (1959), 37--70.

%Make sure to always \/ after slanted stuff, unless the next letter is a . or ,

%Ex 4.2
{\it Explain how to typeset a\/ {\rm roman} word in an italicisez sntence}

{\bf `f'}
{\bf `f\/'}

%RETURN
%EX 4.3
%\def\ic#1{\setbox0=\hbox{#1\/}\dimen0=\wd0
%\setbox0=\hbox{#1}\advance\dimen0 by -\wd0}

{\tenrm asdf}
%Doens't look like a thing anymore

%EX 4.4
%You have to do tenrm ranther than 10rm becuse it does \1 rather than read the whole thing
{\sl sdf}
\def\sl{\it}
{\sl sdf}

%This increases the size directly
\font\ma=cmr5 at 15pt

%This scales the thing.
\font\md=cmr5 scaled 3000

%don't do either though, it donsn't look spectacular.

%\magstep0 is also a thing but it's not that important

{\ma sdffis}
{\md ote}


\end
